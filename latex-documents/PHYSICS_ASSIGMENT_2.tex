%%%%%%
% Short Sectioned Assignment
% LaTeX Template
% Version 1.0 (6/4/2014)
%
% Author: Kiara Navarro
% Base on: Code from Frits Wenneker 
%
% License:
% CC BY-NC-SA 3.0 (http://creativecommons.org/licenses/by-nc-sa/3.0/)
%
%%%%%%

%----------------------------------------------------------------------------------------
%	PACKAGES AND OTHER DOCUMENT CONFIGURATIONS
%----------------------------------------------------------------------------------------

\documentclass[paper=a4, fontsize=11pt]{scrartcl} % A4 paper and 11pt font size
\usepackage[official]{eurosym}
\usepackage[T1]{fontenc} % Use 8-bit encoding that has 256 glyphs
\usepackage[spanish]{babel} % Spanish language/hyphenation
\usepackage{sectsty} % Allows customizing section commands
\allsectionsfont{\centering \normalfont\scshape} % Make all sections centered, the default font and small caps
\PassOptionsToPackage{hyphens}{url}\usepackage{hyperref} % Make link as url
\usepackage{fancyhdr} % Custom headers and footers
\pagestyle{fancyplain} % Makes all pages in the document conform to the custom headers and footers
\fancyhead{} % No page header - if you want one, create it in the same way as the footers below
\fancyfoot[L]{} % Empty left footer
\fancyfoot[C]{} % Empty center footer
\fancyfoot[R]{\thepage} % Page numbering for right footer
\renewcommand{\headrulewidth}{0pt} % Remove header underlines
\renewcommand{\footrulewidth}{0pt} % Remove footer underlines
\setlength{\headheight}{13.6pt} % Customize the height of the header
\setlength\parindent{0pt} % Removes all indentation from paragraphs - comment this line for an assignment with lots of text

%----------------------------------------------------------------------------------------
%	TITLE SECTION
%----------------------------------------------------------------------------------------

\newcommand{\horrule}[1]{\rule{\linewidth}{#1}} % Create horizontal rule command with 1 argument of height

\title{	
\normalfont \normalsize 
\textsc{Universidad Tecnol\'ogica de Panam\'a \\
		Facultad de Ingenier\'ia El\'ectrica\\
		F\'isica II} \\ [25pt] % Your university, school and/or department name(s)
\horrule{0.5pt} \\[0.4cm] % Thin top horizontal rule
\huge Tumor Treating Fields\\ % The assignment title
\horrule{2pt} \\[0.5cm] % Thick bottom horizontal rule
}

\author{Kiara Navarro} % Set your name here

\date{\normalsize Abril 2, 2014} % Today's date or a custom date

\begin{document}

\maketitle % Print the title

\section{Busque en internet, informaci\'on sobre la t\'ecnica empleada y copie la bibliograf\'ia (o infograf\'ia si son enlaces web).(3 documentos distintos)}

La t\'ecnica llamada Tumor Treating Field es una alternativa a los tratamientos convencionales que se tienen contra la lucha del c\'ancer en seres humanos. En vez de hacer uso de agentes quimioterap\'euticos y de radioterapia se utilizan campos electrom\'agneticos de baja intensidad, que no entregan ninguna corriente el\'ectrica al tejido y que a su vez no estimulan nervios o m\'usculos.\\
\\
Consiste en crear un campo electrom\'agnetico que atrae y repele los componentes cargados de las c\'elulas durante la mitosis. Estas fuerzas que los campos EM ejercen logran inducir la muerte de las c\'elulas malignas antes de su divisi\'on.\\
\\
Actualmente esta t\'ecnica se encuentra en dispositivos de la empresa NovocureTMl. Los dispositivos poseen transductores no invasivos que se colocan sobre la piel en la regi\'on que rodea el tumor.  

	\begin{itemize}

		\item Novocure ltd; patent issued for treating a tumor or the like with electric fields at different frequencies. (2012). Journal of Engineering, P\'ag. 1881. \\ 
		Obtenido desde: \url{http://search.proquest.com/docview/1034582036?accountid=34086}
		
		\item NovoTTF therapy treats first recurrent glioblastoma patient in japan. (2013, Aug 02). \\ 
		Obtenido desde: \url{http://search.proquest.com/docview/1460267040?accountid=34086}
		
		\item TTF Therapy: A Drug-Free Treatment for Brain Cancer \\
		\sloppy \url{http://brainworldmagazine.com/ttf-therapy-a-drug-free-treatment-for-brain-cancer/}
		
	\end{itemize}


%------------------------------------------------

\section{\textquestiondown Qu\'e efectos perjudiciales al organismo  humano pueden tener los campos electromagn\'eticos?}
	Investigaciones realizadas explican los campos EM pueden causar da\~nos partes del cuerpo humano si se est\'a expuesto durante un per\'iodo prolongado de tiempo o en base a registros anteriores de exposici\'on.\\
	\\
	El uso frecuente de la telefon\'ia m\'ovil expone al ser humano a  radiaciones de potencia cerca del cerebro incrementando as\'i posibles problemas de tumores cerebrales o dificultades de aprendizaje en ni\~nos. Hoy d\'ia muchas de las tecnolog\'ias que se utilizan est\'an basadas en campos EM. La exposici\'on a HFVT ha demostrado el incremento en riesgo de c\'ancer. 

	\begin{itemize}
		\item Fosburg, M. (2006, Apr). Health effects of electromagnetic radiation. Total Health, Volumen 28. P\'ag: 44-46. \\
		Obtenido desde: \url{http://search.proquest.com/docview/210220769?accountid=34086}
		
		\item De Vocht, F. (2010). "Dirty electricity": What, where, and should we care? Journal of Exposure Science and Environmental Epidemiology, P\'ag: 399-405. \\
		Obtenido desde: \url{http://www.nature.com/jes/journal/v20/n5/full/jes20108a.html}
		
		\item Effects of 60 Hz magnetic fields on teenagers and adults. Environmental Health. \\
		Obtenido desde: \url{http://www.ehjournal.net/content/12/1/42}
	\end{itemize}

%------------------------------------------------

\section{Comparando ambas informaciones, de y sustente en no m\'as de 15 l\'ineas su opini\'on sobre la veracidad de este v\'ideo.}

Este v\'ideo es una muestra de las distintas tecnolog\'ias que se est\'an utilizando para combatir distintos tipos de c\'ancer, a parte de las ya habituales. Una de las principales caracter\'isticas de esta t\'ecnica es que ella no trae consigo los efectos secundarios de operaciones de radioterapia y quimioterapia como lo es la p\'erdida de cabello y apetito, irritaci\'on en la piel, entre otras. Esto hace que la lucha contra la enfermedad entre el paciente que tiene c\'ancer y su familia sea m\'as llevadera. \\

El r\'apido crecimiento de tumores en general se da debido a la frecuente divisi\'on de c\'elulas malignas comparadas con las normales.   Con este proceso se busca eliminar la generaci\'on de calor en los tejidos debido al uso de altas frecuencias. Esto se hace utilizando componentes de baja frecuencia dependiendo del tipo de c\'elula. Lo cierto es que a medida de los a\~nos no se ha tenido una conclusi\'on definida si la t\'ecnica resulta ser mejor que las dem\'as pero se ha podido observar casos en los cuales pacientes tratados anteriormente con tecnolog\'ia de quimioterapia han tenido una mejor respuesta con el uso de TTF prolongando y aumentando su calidad de vida.\\
\\

Inspirado en:
	\begin{itemize}
		\item \sloppy \url {http://www.mountsinai.org/about-us/newsroom/press-releases/head-caps-with-electrodes-may-treat-brain-cancer}

		\item \url {http://www.news-medical.net/news/20111203/University-of-Illinois-Hospital-prescribes-Tumor-Treating-Fields-therapy-for-recurrent-GBM.aspx}

		\item \url{http://www.gizmag.com/treatment-of-brain-tumors-with-electrical-fields/21433/}

		\item \url{http://www.who.int/peh-emf/about/WhatisEMF/en/index1.html}

		\item \url{http://www.betterhealth.vic.gov.au/bhcv2/bhcarticles.nsf/pages/Electromagnetic_radiation_and_health_issues}

		\item \url{http://www.prevention.com/health/healthy-living/electromagnetic-fields-and-your-health}
		
	\end{itemize}
%----------------------------------------------------------------------------------------

\end{document}